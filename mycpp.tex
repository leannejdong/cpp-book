\documentclass{article}
\usepackage[utf8]{inputenc}
\usepackage[T1]{fontenc}
\usepackage[utf8]{inputenc}
\usepackage{lmodern,footnote, url}
\usepackage[a4paper, margin=1in]{geometry}
\usepackage{minted}

\newcommand{\blue}[1]{\textcolor{blue}{#1}}

\large
\title{A review of Bjarne's small book}
\author{Leanne Dong}
\begin{document}
	\maketitle
% \begin{titlepage}
% 	\begin{center}
%     \line(1,0){300}\\
%     [0.65cm]
% 	\huge{\bfseries Assignment II}\\
% 	\line(1,0){300}\\
% 	\textsc{\Large C++ Course Part III}\\
% 	\textsc{\LARGE \today}\\
% 	[5.5cm]
% 	\end{center}
% 	\begin{flushright}
% 		\textsc{\Large T.T. Veldman\\S2754401}\\
% 		[0.5cm]
% 		\textsc{\Large L. Wester\\S2755351}\\
% 		[0.5cm]
% 	\end{flushright}
% \end{titlepage}
\section{The Basics}
\begin{itemize}
	\item Programs
	\item Functions
	\item Types, Variables and Arithmetic
	\item Scope and Lifetime
	\item Constants
	\item Pointers, arrays and References
	\item Tests
	\item Mapping to Hardware
\end{itemize}
\section{User-defined type}
The set of C++ build-in types\footnote{types that can be built from the fundamental types, the const modifier, and the declarator operator} and operations is rich, but deliberately low-level. The directly and efficiently reflect the capabilities of conventional computer hardware. However, they don't provide programmers with high-level facilities to write advanced applications easily. To overcome this, C++ augments the built-in types and operations with a sophisticated set of abstraction mechanisms out of which programmers can build such high-level facilities.

\subsubsection{Strutures}

The first step is building a new type is to putting elements we need into a data structure, a \blue{struct}:

\begin{minted}{c++}
struct Vector {
    int sz;    // number of elements
    double* elem; // pointer to elements
};	
\end{minted}
The first version of \blue{Vector} consists of an \blue{int} and a \blue{double*}. A variable of type \blue{Vector} can be defined as 
\begin{minted}{c++}
	Vector v;
\end{minted}
However, by itself that is not of much use because v's elem pointer doesn't point to anything. For it to be useful, we must give \blue{v} some elements to point to. For instance, construct a \blue{Vector} like
\begin{minted}{c++}
void vector_init(Vector& v, int s)
{
    v.elem = new double[s];
    v.sz = s;
}	
\end{minted}
That is, \blue{v}'s \blue{elem} member gets a pointer produced by the \blue{new} operator, and \blue{v}'s \blue{sz} member gets the number of elements. The \blue{\&} in \blue{Vector\&} indicates that we pass \blue{v} by non-\blue{const} ref, so that \blue{vector\_init()} can modify the vector passed to it.

A simple implementation of \blue{Vector} could be 
\begin{minted}{c++}
double read_and_sum(int s)
// read s integers from cin and return their sum, s is //assumed to be positive
{
    Vector v;
    vector_init(v, s); // allocate s elements for v
    for (int i = 0; i!=s; ++i)
    {
        std::cin >> v.elem[i]; // read into elements
    }
    double sum = 0;
    for (int i = 0; i!=s; ++i)
    {
        sum+=v.elem[i];
    }
    std::cout << sum << " \n";
    return sum;
}

int main()
{
    read_and_sum(10);
}	
\end{minted}
There was a long to go from above to the elegant \blue{std::vector}.

Comiler Explorer: \url{https://godbolt.org/z/Esnr8E}
\\

Now another example from Herb's talk.
\begin{minted}{c++}
#include <iostream>
#include <vector>
#include <string>
#include <algorithm>

struct User {
    std::string name;
    int age;
};

std::vector<User> users = { {"Cat", 3}, {"Fish", 5} };

int main()
{
    auto sort_by_age = [] (auto& lhs, auto& rhs)
    {
        return lhs.age < rhs.age;
    };

    std::sort(users.begin(), users.end(), sort_by_age);
}
\end{minted}
Compiler explorer: \url{https://godbolt.org/z/noea68}

\subsubsection{Classes}

\subsubsection{Unions}

\subsubsection{Unions}

\subsubsection{Enumerations}
% \section*{Exercise 1}
% \subsection*{main.cpp}
% \begin{minted}[frame=lines, linenos, fontsize=\large]
% {c++}
% #include "header.ih"
%
% int main()
% {
%     fn1();
%     fn2();
% }
% \end{minted}


\end{document}











