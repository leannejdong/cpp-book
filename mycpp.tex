\documentclass{article}
\usepackage[utf8]{inputenc}
\usepackage[T1]{fontenc}
\usepackage[utf8]{inputenc}
\usepackage{lmodern,footnote, url, hyperref}
\hypersetup{
	colorlinks,
	citecolor = red,
	filecolor = violet,
	linkcolor = blue,
	urlcolor = violet,
	linktoc = all,
}
\usepackage[a4paper, margin=1in]{geometry}
\usepackage{minted}

\newcommand{\be}[1]{\textcolor{blue}{#1}}

\large
\title{A review of Bjarne's small book}
\author{Leanne Dong}
\begin{document}
	\maketitle
	\tableofcontents
\section{The Basics}
\begin{itemize}
	\item Programs
	\item Functions and control flows
	\item Types, Variables and Arithmetic
	\item Scope and Lifetime
	\item Constants
	\item Pointers, arrays and References
	\item Tests
	\item Mapping to Hardware
\end{itemize}

\href{https://gcc.godbolt.org/z/bah4zM}{Iteration}
\section{User-defined type}
The set of C++ build-in types\footnote{types that can be built from the fundamental types, the const modifier, and the declarator operator} and operations is rich, but deliberately low-level. The directly and efficiently reflect the capabilities of conventional computer hardware. However, they don't provide programmers with high-level facilities to write advanced applications easily. To overcome this, C++ augments the built-in types and operations with a sophisticated set of abstraction mechanisms out of which programmers can build such high-level facilities.

\subsubsection{Structures}

The first step is building a new type is to putting elements we need into a data structure, a \be{struct}:

\begin{minted}{c++}
struct Vector {
    int sz;    // number of elements
    double* elem; // pointer to elements
};	
\end{minted}
The first version of \be{Vector} consists of an \be{int} and a \be{double*}. A variable of type \be{Vector} can be defined as 
\begin{minted}{c++}
	Vector v;
\end{minted}
However, by itself that is not of much use because v's elem pointer doesn't point to anything. For it to be useful, we must give \be{v} some elements to point to. For instance, construct a \be{Vector} like
\begin{minted}{c++}
void vector_init(Vector& v, int s)
{
    v.elem = new double[s];
    v.sz = s;
}	
\end{minted}
That is, \be{v}'s \be{elem} member gets a pointer produced by the \be{new} operator, and \be{v}'s \be{sz} member gets the number of elements. The \be{\&} in \be{Vector\&} indicates that we pass \be{v} by non-\be{const} ref, so that \be{vector\_init()} can modify the vector passed to it.

The \be{new} operator allocates memory from \emph{free store} (also known as \emph{dynamic memory, or heap} ).
Object allocated on the free store are independent of the scopt from which they are created and live until they are destroyed using the \be{delete} operator.

A simple implementation of \be{Vector} could be 
\begin{minted}{c++}
double read_and_sum(int s)
// read s integers from cin and return their sum, s is //assumed to be positive
{
    Vector v;
    vector_init(v, s); // allocate s elements for v
    for (int i = 0; i!=s; ++i)
    {
        std::cin >> v.elem[i]; // read into elements
    }
    double sum = 0;
    for (int i = 0; i!=s; ++i)
    {
        sum+=v.elem[i];
    }
    std::cout << sum << " \n";
    return sum;
}

int main()
{
    read_and_sum(10);
}	
\end{minted}
There was a long to go from above to the elegant \be{std::vector}.

\href{https://godbolt.org/z/vjEMeP}{Compiler Explorer}

We use \be{.}(dot) to access \be{struct} members through a name (and through a reference) and \be{->} to access 
\be{struct} members through a pointer. For instance

\begin{minted}{c++}
void f(Vector v, Vector& rv, Vector* pv)
{
    int i1 = v.sz; // access through name
    int i2 = rv.sz; // access through reference
    int i3 = pv->sz; // access through pointer
}
\end{minted}


Now another example from Herb's talk.
\begin{minted}{c++}
#include <iostream>
#include <vector>
#include <string>
#include <algorithm>

struct User {
    std::string name;
    int age;
};

std::vector<User> users = { {"Cat", 3}, {"Fish", 5} };

int main()
{
    auto sort_by_age = [] (auto& lhs, auto& rhs)
    {
        return lhs.age < rhs.age;
    };

    std::sort(users.begin(), users.end(), sort_by_age);
}
\end{minted}
\href{https://godbolt.org/z/noea68}{Compiler Explorer}

\subsubsection{Classes}
From the \be{struct} example we see that having the data specified separately from the operations on it 
has advantages. Now we will establish a tighter connection between the representation and the operations for a user-defined type to have all the properties of a `real type'. Specifically, we should keep representation inaccessible to users so as to ease use, ensure consistent use of the data, and allow us to later improve the representation. To this end, we need to distinguish
\begin{enumerate}
	\item The interface
	\item A type (to be used by all)
	\item The implementation of the type (which has access to the otherwise inaccessible data)
\end{enumerate}
The is the so-called class. Class provides greater level of abstractions compared to struct.The interface is defined by the \be{public} member of a class, and \be{private} members are accessible only through that interface. For instance,
\begin{minted}{c++}
class Vector {
public:
Vector(int s) :elem{new double[s]}, sz(s) { } //construct a vector
double& operator[] (int i) { return elem[i]; } //element access via subscripting
int size() { return sz; }

private:
double* elem; // the pointers to the elements
int sz;       // the number of elements	
}
\end{minted}
Let us now define a variable of our new type \be{Vector}:
\begin{minted}{c++}
	Vector v(6); // a Vector with 6 elements
\end{minted}
Intuitively, the \be{Vector} object is a placeholder containing a pointer to the elements (\be{elem}) and the number of elements (\be{sz}). The number of elements can vary from \be{Vector} object to \be{Vector} object, and one particular \be{Vector} object can have a different number of elements at different times which we shall see soon. However, the \be{Vector} object itself is always the same size.\footnote{There are two different meaning of `size' here, the first one is the size indicated by the \be{sz} member (\be{v.sz}), which can vary; The second one is the size of the Vector type (\be{sizeof(v)}), i.e. the size of pointer and an integer.\href{https://godbolt.org/z/PM6437}{example 1} and \href{https://godbolt.org/z/GsW74s}{example 2}. Note in the 2nd example that, \be{sizeof(V3)} is different from what one might expect, due to the padding introduced for the purpose of the alignment. Note also that, do not assume that the size of a type is just the sum of the sizes of its members.} This is the fundamental building blocks for handling varying amounts of information in C++. Later on, we will discuss the design and usage of such objects.

Here, the representation of a \be{Vector} (the member variables \be{elem} and \be{sz}) is accessible only through the interface provided by public members: \be{Vector()}, \be{operator[]()} and \be{size()}. The \be{read\_and\_sum()} example introduced earlier simplifies to 
\begin{minted}{c++}
double read_and_sum(int s)
// read s integers from cin and return their sum, s is assumed to be positive
{
    Vector v(6); // make a vector of s elements
    for (int i = 0; i!=v.size(); ++i)
    {
        std::cin >> v[i]; // read into elements
    }
    double sum = 0;
    for (int i = 0; i!=s; ++i)
    {
        sum+=v[i];
    }
    std::cout << sum << " \n";
    return sum;
}	
\end{minted}
A member ``function'' with the same name as its class is called a \emph{constructor}, that is, a function used to construct objects of a class. So, the constructor \be{Vector()} replaces \be{vector\_init()} earlier. Unlike an ordinary function, a constructor is guaranteed to be used to initialize objects of its class. Hence, defining a constructor eliminates the problem of uninitialized variables for a class.

\be{Vector(int)} defines how objects of type \be{Vector} are constructed by explicitly stating that it needs an integer to do so. The integer is used as the number of elements. The constructor initializes the \be{Vector} members using a member initializer list:
\begin{minted}{c++}
	:elem{new double[s]}, sz(s)
\end{minted}
Here it says, we first initialize \be{elem} with a pointer to \be{s} elements of type \be{double} obtained from the free store. Then we initialize \be{sz} to \be{s}. The subscript function \be{operator[]} allows us to access elements. It returns a reference to the appropriate element (a \be{double\&} allowing both reading and writing).

The \be{size()} function provides the users with the number of elements.

Clearly, several aspects are missing.
\begin{enumerate}
	\item Error handling
	\item A lack of mechanism to return the array of doubles acquired by \be{new}. Shortly, we will show how to use \be{destructor} does this.
\end{enumerate}
There is no fundamental difference between a \be{struct} and a \be{class}; a \be{struct} is simply a \be{class} with members \be{public} by default. For instance, one can define constructors and member functions inside \be{struct}.
 The implementation of the discussed example is at 
\href{https://godbolt.org/z/E6M64s}{Compiler Explorer}
\subsubsection{Unions}


\subsubsection{Enumerations}
% \section*{Exercise 1}
% \subsection*{main.cpp}
% \begin{minted}[frame=lines, linenos, fontsize=\large]
% {c++}
% #include "header.ih"
%
% int main()
% {
%     fn1();
%     fn2();
% }
% \end{minted}

\section{Modularity}

\section{Classes}

\section{Essential Operations}

\section{Templates}

\subsection{Lambda}
Lambdas come up when we would like to write programs that utilize unnamed function objects. The following snippet shows how a lambda is defined in C++ source code. The syntax for a lambda is as follows:
\begin{minted}{c++}
	[](){};
\end{minted}
The braces represent the capture block. A lambda uses a block to capture existing variables to be used in the lambda. 
\section{Concepts and Generic Programming}

\section{Library Overview}

\section{String and Regular Expression}

\section{Input and Output}

\section{Containers}

\section{Algorithms}

\section{Utilities}

\section{Numerics}

\section{Concurrency}

\section{History and Compatibility}

\end{document}











